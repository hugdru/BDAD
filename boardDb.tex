\documentclass[a4paper]{article}

\usepackage[portuguese]{babel}
\usepackage[utf8]{inputenc}
\usepackage{indentfirst}
\usepackage{graphicx}
\usepackage{verbatim}

\begin{document}

\setlength{\textwidth}{16cm}
\setlength{\textheight}{22cm}

\title{\Huge\textbf{Jogos de tabuleiro}\linebreak\linebreak\linebreak
\Large\textbf{Relatório 1ª Fase}\linebreak\linebreak
\includegraphics[height=6cm, width=7cm]{feup.pdf}\linebreak \linebreak
\Large{Mestrado Integrado em Engenharia Informática e Computação} \linebreak \linebreak
\Large{Base de dados}\linebreak
}

\author{\textbf{Grupo 601:}\\ Hugo Ari Rodrigues Drumond --- 201102900 \\  Ricardo Jorge Matos Figueiredo --- 201100687\\\linebreak\linebreak \\
 \\ Faculdade de Engenharia da Universidade do Porto \\ Rua Roberto Frias, 4200--65 Porto, Portugal \linebreak\linebreak\linebreak
\linebreak\linebreak\vspace{1cm}}
\maketitle
\thispagestyle{empty}

\newpage

\section{Resumo}
%Decidimos elaborar uma base de dados para uma Empresa organizadora de \textit{Jogos de tabuleiro}.    \cite{creator}


%Descrever muito sumariamente (1-2 parágrafos) o trabalho que está a ser reportado

%\section{Introdução}

%Descrever os objectivos e motivação do trabalho.
%Todas as figuras devem ser referidas no texto. %\ref{fig:codigoFigura}

\section{Contexto}
Esta base de dados destina-se a um Salão de jogos que organiza jogos de Tabuleiro e foi desenvolvida, inicialmente, pensando num só jogo, o xadrez. Posteriormente, então, foi feita uma generalização. Em suma, são guardados os dados dos jogadores e dos torneios de uma dada temporada. Isto, possibilitará aos utilizadores da base de dados: rigor ao planear eventos, versatilidade, facilidade de registo, entre outros. Por exemplo, se eu quisesse organizar um torneio, em condições, iria ter saber à priori: os escalões, o jogo a que diz respeito, a temporada, as equipas inscritas e os patrocinadores. E registá-los em algum sítio para que depois possa associá-los, direta ou indiretamente, a uma partida e uma partida a equipas. A nossa base de dados tem como único propósito tornar esse \{pré,pós\}registo trivial.
%Para quem se destina, quem e que vai usar, 

\section{Conceitos Principais}
%Identificar as classes e falar um pouco sobre elas. As óbvias não é necessário.
dasdasd
dasdasda
dadosdasda

\section{Diagrama de classes UML}
%Incluir todas as restrições, digitalizar o diagrama ou fazer num programa qualquer. Incluir notas.

\begin{center}
  \includegraphics[scale=0.83]{BDAD_DIAGRAMA.jpeg}
\end{center}

\clearpage
\addcontentsline{toc}{section}{Bibliografia}
\renewcommand\refname{Bibliografia}
\bibliographystyle{plain}
\bibliography{myRefs}

\end{document}
