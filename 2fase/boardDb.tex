\documentclass[a4paper]{article}

\usepackage[portuguese]{babel}
\usepackage[utf8]{inputenc}
\usepackage{indentfirst}
\usepackage{graphicx}
\usepackage{verbatim}
\usepackage[T1]{fontenc}

\begin{document}

\setlength{\textwidth}{16cm}
\setlength{\textheight}{22cm}

\title{\Huge\textbf{Jogos de tabuleiro}\linebreak\linebreak\linebreak
\Large\textbf{Relatório 2ª Fase}\linebreak\linebreak
\includegraphics[height=6cm, width=7cm]{feup.pdf}\linebreak \linebreak
\Large{Mestrado Integrado em Engenharia Informática e Computação} \linebreak \linebreak
\Large{Base de dados}\linebreak
}

\author{\textbf{Grupo 601:}\\ Hugo Ari Rodrigues Drumond --- 201102900 \\  Ricardo Jorge Matos Figueiredo --- 201100687 \\ Gustavo Assis Freitas --- 200602187\\\linebreak\linebreak \\
 \\ Faculdade de Engenharia da Universidade do Porto \\ Rua Roberto Frias, 4200--65 Porto, Portugal \linebreak\linebreak\linebreak
\linebreak\linebreak\vspace{1cm}}
\maketitle
\thispagestyle{empty}

\newpage

%Decidimos elaborar uma base de dados para uma Empresa organizadora de \textit{Jogos de tabuleiro}.    \cite{creator}


%Descrever muito sumariamente (1-2 parágrafos) o trabalho que está a ser reportado

%\section{Introdução}

%Descrever os objectivos e motivação do trabalho.
%Todas as figuras devem ser referidas no texto. %\ref{fig:codigoFigura}

\section{Contexto}
Esta base de dados destina-se a um Salão de jogos que organiza jogos de Tabuleiro e foi desenvolvida, inicialmente, pensando num só jogo, o xadrez. Posteriormente, então, foi feita uma generalização. Em suma, são guardados os dados dos jogadores e dos torneios de uma dada temporada. Isto, possibilitará aos utilizadores da base de dados: rigor ao planear eventos, versatilidade, facilidade de registo, entre outros. Por exemplo, se eu quisesse organizar um torneio, em condições, iria ter saber à priori: os escalões, o jogo a que diz respeito, a temporada, as equipas inscritas e os patrocinadores. E registá-los em algum sítio para que depois possa associá-los, direta ou indiretamente, a uma partida e uma partida a equipas. A nossa base de dados tem como único propósito tornar esse \{pré,pós\}registo trivial.

%explicar de forma legível o nosso diagrama de classes uml melhorado. Tal como é feito nos primeiros exercícios.
\section{Conceitos Principais}
Num torneio de jogos de tabuleiro podem haver várias partidas entre duas ou mais equipas num dado escalão de um torneio. As partidas podem ocorrer em diversos sítios, têm de ser reguladas por um ou mais árbitros qualificados para o jogo em disputa e os resultados devem ser guardados. Cada equipa é formada por um ou mais jogadores.

%explicar como fizemos o mapeamento para o modelo relacional e qual o nível de normalização.
\section{Passagem ao modelo relacional e normalização}
A transição do diagrama de classes para o schema da base de dados foi feita sem grandes problemas. Optámos por separar a generalização em duas tabelas segundo o estilo orientado a objetos, porque ao fazer isto ficamos entre o ponto de equilibro entre poupar nas ligações de tabelas e nos recursos de armazenamento. Uma relação, menos ligações embora haja desperdício de armazenamento; Três relações, mais ligações e mesmo uso de recursos de armazenamento. A única ternária no nosso uml foi decomposta em quatro tabelas: Equipa, Patrocinador, Torneio, EquipaPatrocinadorTorneio. A classe de associação sofreu uma transformação idêntica à ternária: Equipa, Partida, EquipaPartida.
\\\newline
Database Schema:\\
Jogador( \underline{idJogador}, codigoPostal, dataNascimento, numeroAndar, rua, telefone, idPais->Pais, idCidade->Cidade, idExtensao->Extensao, email) \\
Equipa( \underline{nome}, abreviatura) \\ %dúvida aqui sobre bcnf
JogadorEquipa( \underline{idEquipa->Equipa}, \underline{idJogador->Jogador} ) \\
Árbitro( \underline{idÁrbitro}, idPais->Pais, idCidade->Cidade, idExtensao->Extensao, observacoes ) \\
LocalEncontro( \underline{idLocalEncontro}, idCidade->Cidade, idExtensao->Extensao, codigoPostal, rua, telefone ) \\
Cidade( \underline{idCidade}, nome, idPais->Pais ) \\
TipoJogo( \underline{idTipoJogo}, nome ) \\
ArbitroTipoJogo( \underline{idArbitro->Arbitro}, \underline{idTipoJogo->TipoJogo} ) \\
Partida( \underline{idPartida}, dataInicio, duracao, idEscalao->Escalao ) \\
ArbitroPartida( \underline{idArbitro->Arbitro}, \underline{idPartida->Partida} \\
Escalao( \underline{idEscalao}, nome ) \\
EquipaPartida( \underline{idEquipa->Equipa}, \underline{idPartida->Partida}, posicao, resultado ) \\
Patrocinador( \underline{idPatrocinador->Patrocinador}, nome ) \\
EquipaPatrocinadorTorneio( \underline{idEquipa->Equipa}, \underline{idPatrocinador->Patrocinador}, \underline{idTorneio->Torneio} ) \\

%indicar o tipo de restrições que foram usadas e a criação das tabelas
\section{Linguagem de Definição de dados e Restrições}
Na nossa base de dados existem algumas restrições e gatilhos.
Restrições de valor: NOT NULL, limites de atributo.
Restrições de atributos: ainda não temos, temos de arranjar maneira de colocar isto em qualquer lado
Asserção: arranjar uma maneira de colocar isto
E obviamente que foram definidas chaves primárias e estrangeiras que são um tipo de restrição.

%Dizer o que irá ser inserido na bases de dados
\section{Linguagem de Manipulação de dados}

%Incluir todas as restrições, digitalizar o diagrama ou fazer num programa qualquer. Incluir notas.
\section{Diagrama de classes UML melhorado}

\end{document}
